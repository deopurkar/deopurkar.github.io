% Created 2021-10-05 Tue 10:07
% Intended LaTeX compiler: pdflatex
\documentclass[11pt]{article}
\usepackage[utf8]{inputenc}
\usepackage[T1]{fontenc}
\usepackage{graphicx}
\usepackage{grffile}
\usepackage{longtable}
\usepackage{wrapfig}
\usepackage{rotating}
\usepackage[normalem]{ulem}
\usepackage{amsmath}
\usepackage{textcomp}
\usepackage{amssymb}
\usepackage{capt-of}
\usepackage{hyperref}
\usepackage{amsmath}
\usepackage{amssymb}
\usepackage{amsthm}
\usepackage{thmtools}
\usepackage[margin=2cm]{geometry}
\usepackage{titlesec}
\usepackage{mdframed}
\usepackage{xcolor}
\newmdenv[linecolor=red, linewidth=1pt, topline=false, bottomline=false, backgroundcolor = red!10!white]{skipped}
\newmdenv[linecolor=green, topline=false, bottomline=false, backgroundcolor = green!10!white]{student}
\titleformat{\subsubsection}[runin]{\bfseries}{\thesubsubsection}{1em}{}
% Let us keep this minimial
% Let us also define things only if they are previously undefined.

% Common theorem-like environments
\ifcsname theorem\endcsname{}\else\declaretheorem[parent=section]{theorem}\fi
\ifcsname corollary\endcsname{}\else\declaretheorem[sibling=theorem]{corollary}\fi
\ifcsname lemma\endcsname{}\else\declaretheorem[sibling=theorem]{lemma}\fi
\ifcsname proposition\endcsname{}\else\declaretheorem[sibling=theorem]{proposition}\fi
\ifcsname conjecture\endcsname{}\else\declaretheorem[sibling=theorem]{conjecture}\fi
\ifcsname problem\endcsname{}\else\declaretheorem[sibling=theorem]{problem}\fi
\ifcsname question\endcsname{}\else\declaretheorem[sibling=theorem]{question}\fi
\ifcsname definition\endcsname{}\else\declaretheorem[sibling=theorem, style=definition]{definition}\fi
\ifcsname exercise\endcsname{}\else\declaretheorem[sibling=theorem, style=definition]{exercise}\fi
\ifcsname example\endcsname{}\else\declaretheorem[sibling=theorem, style=definition]{example}\fi
\ifcsname remark\endcsname{}\else\declaretheorem[sibling=theorem, style=remark]{remark}\fi

% Common abbreviations

% Absolutely standard rings and fields
\providecommand {\N}{{\bf N}}
\providecommand {\Z}{{\bf Z}}
\providecommand {\Q}{{\bf Q}}
\providecommand {\R}{{\bf R}}
\providecommand {\C}{{\bf C}}

% Common spaces grassmannian
\renewcommand {\P}{{\bf P}}
\providecommand {\Gr}{{\bf Gr}}
\providecommand {\A}{{\bf A}}

% Groups
\providecommand{\SL}{\operatorname{SL}}
\providecommand{\GL}{\operatorname{GL}}
\providecommand{\PGL}{\operatorname{PGL}}
\providecommand{\Gm}{{\bf G}_m}

% f \from G \to H reads much better than f \colon G \to H
\providecommand {\from}{{\colon}}

% Absolutely standard notation
\providecommand{\spec}{\operatorname{Spec}}
\providecommand{\proj}{\operatorname{Proj}}
\providecommand{\coker}{\operatorname{coker}}
% Kernel is already defined
\providecommand{\Blowup}{\operatorname{Bl}}
\providecommand{\Hom}{\operatorname{Hom}}
\providecommand{\Ext}{\operatorname{Ext}}
\providecommand{\Tor}{\operatorname{Tor}}
\providecommand{\End}{\operatorname{End}}
\providecommand{\Aut}{\operatorname{Aut}}
\providecommand{\codim}{\operatorname{codim}}
% Dim is already defined
\providecommand{\Pic}{\operatorname{Pic}}
\providecommand{\Sym}{\operatorname{Sym}}
\providecommand{\rk}{\operatorname{rk}}
\author{Anand Deopurkar}
\date{\today}
\title{Advanced topics in algebra: Algebraic geometry}
\hypersetup{
 pdfauthor={Anand Deopurkar},
 pdftitle={Advanced topics in algebra: Algebraic geometry},
 pdfkeywords={},
 pdfsubject={},
 pdfcreator={Emacs 27.1 (Org mode 9.4.6)}, 
 pdflang={English}}
\begin{document}

\maketitle
This year, the course will be on algebraic geometry, the study of algebraic equations using geometry and geometric shapes using algebra.
We will begin with affine varieties---solution sets of polynomial equations---and use these as building blocks to construct a rich world of geometric objects that include Riemann surfaces, projective spaces, Grassmannians, etc.

\section{Announcements}
\label{sec:orgd00a6ab}
\begin{enumerate}
\item Homework 3 is up on Gradescope.
\item The \href{classwork09.pdf}{classwork for week 9} is up.
\end{enumerate}

\section{Links}
\label{sec:org5370edf}
\begin{itemize}
\item Lectures: 1 (Monday 10--11, Moran G007), 2 (Wed 10--11, Moran G008), 3 (Thu 11--12, HN 4.41), 4 (Fri 12--13, Moran G007)
\item Zoom lecture: \url{https://anu.zoom.us/j/87920733031} (you know the password)
\item Zoom office hours: Thursday 10--11 (same room as the lecture), continuing into Thursday's class.
\item Discussion forum: \url{https://ag2021.zulipchat.com}.
\item The assignments will be due on Gradescope.
\end{itemize}

\section{References}
\label{sec:orgd1314c0}
\begin{enumerate}
\item \href{notes}{Our own notes}
\item \href{https://link.springer.com/book/10.1007/978-3-642-37956-7}{Basic Algebraic Geometry, Part I by I. Shafarevich}
\item \href{https://www.mathematik.uni-kl.de/\~gathmann/class/alggeom-2002/alggeom-2002.pdf}{Online notes by A. Gathmann}
\end{enumerate}
\subsection{Resources for mathematical writing}
\label{sec:orga63b5ce}
\begin{enumerate}
\item \href{https://www.math.ucla.edu/\~pak/papers/how-to-write1.pdf}{How to write a clear math paper by Igor Pak.}
\item \href{https://jmlr.csail.mit.edu/reviewing-papers/knuth\_mathematical\_writing.pdf}{Mathematical Writing by Donald Knuth et al.}
\end{enumerate}

\section{Plan}
\label{sec:orgee0187c}
This is a tentative outline of the course, subject to change.
\begin{center}
\begin{tabular}{rll}
\hline
Week & Topic & Assessment\\
\hline
1 & What is algebraic geometry? The first examples: algebraic subsets of affine space. Lecture 1: \href{https://web.microsoftstream.com/video/cf234444-df4b-4b65-8016-a3c1b7539891?channelId=cd4289e5-e630-458c-8ea0-2bd2632faea0}{Video}/\href{notes/2021-07-27.pdf}{Board}, Lecture 2: \href{https://web.microsoftstream.com/video/9baf2139-0fa8-4419-89f2-aabff250de07?channelId=cd4289e5-e630-458c-8ea0-2bd2632faea0}{Video}/\href{notes/2021-07-08.pdf}{Board} & \\
2 & \href{classwork02.pdf}{The Nullstellensatz}: Lecture 1: \href{https://web.microsoftstream.com/video/307216ea-46a1-40dd-9389-7bde7ea8b439?channelId=cd4289e5-e630-458c-8ea0-2bd2632faea0}{Video}/\href{notes/2021-08-02.pdf}{Board}, Lecture 2: \href{https://web.microsoftstream.com/video/a3127e92-540d-47a2-8c8f-66e8e2411163?list=studio}{Video}/\href{notes/2021-08-06.pdf}{Board} & \\
3 & \href{classwork03.pdf}{Regular functions and regular maps}: Lecture 1: \href{https://web.microsoftstream.com/video/ca9210bf-b61b-4b1f-a077-8a78c29b404e}{Video}/\href{notes/2021-08-09.pdf}{Board}, Lecture 2: \href{https://web.microsoftstream.com/video/048c9983-30be-400e-9e1b-a6019c369b83}{Video}/\href{notes/2021-08-13.pdf}{Board} & Assignment 1\\
4 & \href{classwork04.pdf}{General algebraic varieties}: Lecture 1: \href{https://web.microsoftstream.com/video/aff3b45e-5fbd-44e7-b655-1e19277560f9}{Video}/\href{notes/2021-08-16.pdf}{Board}, Lecture 2: \href{https://web.microsoftstream.com/video/34028770-b503-41f0-95f4-20377a1cc55a?list=studio}{Video}/\href{notes/2021-08-20.pdf}{Board} & \\
5 & \href{classwork05.pdf}{Functions and maps on algebraic varieties}: Lecture 1: \href{https://web.microsoftstream.com/video/248369c8-6e61-4746-a301-eafc8c7b8853}{Video}/\href{notes/2021-08-23.pdf}{Board}, Lecture 2: \href{https://web.microsoftstream.com/video/8efe045b-4d50-4e82-bb62-2042fa7678c9}{Video}/\href{notes/2021-08-27.pdf}{Board} & Assignment 2\\
6 & \href{classwork06.pdf}{Products and the Segre embedding}: Lecture 1: \href{https://web.microsoftstream.com/video/8ce2960f-c731-4001-9772-2bc88f34d8fa}{Video}/\href{notes/2021-08-30.pdf}{Board}, Lecture 2: \href{https://web.microsoftstream.com/video/f5f36337-3a49-4adf-84f6-87d0ef80229d}{Video}/\href{notes/2021-09-03.pdf}{Board} & \\
7 & \href{classwork07.pdf}{Grassmannians}: Lecture 1: \href{https://web.microsoftstream.com/video/515cea51-64d5-445d-aaaa-6526765b021c?list=studio}{Video}, Lecture 2: \href{https://web.microsoftstream.com/video/beb6fe34-c764-4c09-92c2-93edd85f7861}{Video}/\href{notes/2021-09-24.pdf}{Board} & Mid-semester exam\\
8 & \href{classwork08.pdf}{Irreducibility and rational maps}: Lecture 1: \href{https://web.microsoftstream.com/video/5bb2a527-c57a-4ef4-a3dd-0106216db85f}{Video}/\href{notes/2021-09-27.pdf}{Board}, Lecture 2: \href{https://web.microsoftstream.com/video/09931dad-b242-415a-b602-f73895fb437e}{Video}/\href{notes/2021-10-01.pdf}{Board} & Assignment 3\\
9 & \href{classwork09.pdf}{Dimension} & \\
10 & Tangent spaces and smoothness & Assignment 4\\
11 & Completeness & \\
12 & ??? & Assignment 5\\
\hline
\end{tabular}
\end{center}

\section{Pre-requisites}
\label{sec:orgca337a4}
Algebra 1 and Algebra 2.

Algebraic geometry interacts deeply with many other areas of mathematics, so some background in topology, complex analysis, and differential geometry will be helpful, but not required. 

\section{Assessment}
\label{sec:org160f9aa}
The final mark is based on three factors.
\begin{enumerate}
\item (50\%) Homework + classwork
\item (25\%) A mid-semester exam (take-home, in-person, or Zoom)
\item (25\%) A final exam (take-home, in-person, or Zoom)
\end{enumerate}
The classwork will consist of progress reports and the associated write-up, each equivalent to 1 homework set.
If you are a masters student or taking the course as an ASC/ASE, your assessment will include an additional project component.

\subsection{Classwork}
\label{sec:org77d74b8}
The format of our classwork will be as follows. The first meeting of the week (Monday) will be a traditional lecture in which I will explain the key ideas of a topic, but not prove all the details. In the following two meetings (Wednesday, Thursday), we will work out the details in small groups. By the end of Thursday, you will write a ``progress report'' (a 1-2 sentence summary of your progress for each question) and submit it on Gradescope. On Friday, based on the progress reports, I will present the solutions. Students (selected at random) will write up the solutions and send them to me in a week. I will add the write-ups to \href{notes}{our notes}. 

\section{Policies}
\label{sec:orgaef2818}
\subsection{Late submission}
\label{sec:orgb0b89ca}
I will not accept any late submissions, except for medical emergencies with a medical certificate.
To compensate for this strict policy, I will drop the lowest among the (homework+classwork) marks.

\subsection{Collaboration}
\label{sec:orgef6576a}
You are allowed, even encouraged, to work with others on homework assignments, but you must write up your solutions \textbf{on your own}. In other words, you \textbf{may not} copy someone else's write-up and you \textbf{may not} write your solutions side by side with someone else. On your submission, you must write the names of your collaborators. This is a matter of academic honesty; it will not affect your marks. 

There will be no collaboration on the exams.

\section{Miscellenous}
\label{sec:org11b8f52}
<div class=``intro''>
\begin{quote}
Algebra is nothing but written geometry; geometry is nothing but pictured algebra.  
--- Sophie Germain (1776--1831)
\end{quote}
\begin{quote}
No attention should be paid to the fact that algebra and geometry are different in appearance.  
--- Omar Khayyam (1048--1131)
\end{quote}
</div>
\end{document}