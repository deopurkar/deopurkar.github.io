% Created 2021-07-27 Tue 13:12
% Intended LaTeX compiler: pdflatex
\documentclass[11pt]{article}
\usepackage[utf8]{inputenc}
\usepackage[T1]{fontenc}
\usepackage{graphicx}
\usepackage{grffile}
\usepackage{longtable}
\usepackage{wrapfig}
\usepackage{rotating}
\usepackage[normalem]{ulem}
\usepackage{amsmath}
\usepackage{textcomp}
\usepackage{amssymb}
\usepackage{capt-of}
\usepackage{hyperref}
\usepackage{amsmath}
\usepackage{amssymb}
\usepackage{amsthm}
\usepackage{thmtools}
\usepackage{geometry}
\usepackage{titlesec}
\usepackage{mdframed}
\usepackage{xcolor}
\newmdenv[linecolor=red, linewidth=1pt, topline=false, bottomline=false, backgroundcolor = red!10!white]{skipped}
\titleformat{\subsubsection}[runin]{\bfseries}{\thesubsubsection}{1em}{}
% Let us keep this minimial
% Let us also define things only if they are previously undefined.

% Common theorem-like environments
\ifcsname theorem\endcsname{}\else\declaretheorem[parent=section]{theorem}\fi
\ifcsname corollary\endcsname{}\else\declaretheorem[sibling=theorem]{corollary}\fi
\ifcsname lemma\endcsname{}\else\declaretheorem[sibling=theorem]{lemma}\fi
\ifcsname proposition\endcsname{}\else\declaretheorem[sibling=theorem]{proposition}\fi
\ifcsname conjecture\endcsname{}\else\declaretheorem[sibling=theorem]{conjecture}\fi
\ifcsname problem\endcsname{}\else\declaretheorem[sibling=theorem]{problem}\fi
\ifcsname question\endcsname{}\else\declaretheorem[sibling=theorem]{question}\fi
\ifcsname definition\endcsname{}\else\declaretheorem[sibling=theorem, style=definition]{definition}\fi
\ifcsname exercise\endcsname{}\else\declaretheorem[sibling=theorem, style=definition]{exercise}\fi
\ifcsname example\endcsname{}\else\declaretheorem[sibling=theorem, style=definition]{example}\fi
\ifcsname remark\endcsname{}\else\declaretheorem[sibling=theorem, style=remark]{remark}\fi

% Common abbreviations

% Absolutely standard rings and fields
\providecommand {\N}{{\bf N}}
\providecommand {\Z}{{\bf Z}}
\providecommand {\Q}{{\bf Q}}
\providecommand {\R}{{\bf R}}
\providecommand {\C}{{\bf C}}

% Common spaces grassmannian
\renewcommand {\P}{{\bf P}}
\providecommand {\Gr}{{\bf Gr}}
\providecommand {\A}{{\bf A}}

% Groups
\providecommand{\SL}{\operatorname{SL}}
\providecommand{\GL}{\operatorname{GL}}
\providecommand{\PGL}{\operatorname{PGL}}
\providecommand{\Gm}{{\bf G}_m}

% f \from G \to H reads much better than f \colon G \to H
\providecommand {\from}{{\colon}}

% Absolutely standard notation
\providecommand{\spec}{\operatorname{Spec}}
\providecommand{\proj}{\operatorname{Proj}}
\providecommand{\coker}{\operatorname{coker}}
% Kernel is already defined
\providecommand{\Blowup}{\operatorname{Bl}}
\providecommand{\Hom}{\operatorname{Hom}}
\providecommand{\Ext}{\operatorname{Ext}}
\providecommand{\Tor}{\operatorname{Tor}}
\providecommand{\End}{\operatorname{End}}
\providecommand{\Aut}{\operatorname{Aut}}
\providecommand{\codim}{\operatorname{codim}}
% Dim is already defined
\providecommand{\Pic}{\operatorname{Pic}}
\providecommand{\Sym}{\operatorname{Sym}}
\providecommand{\rk}{\operatorname{rk}}
\setcounter{secnumdepth}{3}
\author{Anand Deopurkar}
\date{\today}
\title{Algebraic geometry (Notes)}
\hypersetup{
 pdfauthor={Anand Deopurkar},
 pdftitle={Algebraic geometry (Notes)},
 pdfkeywords={},
 pdfsubject={},
 pdfcreator={Emacs 26.3 (Org mode 9.4.4)}, 
 pdflang={English}}
\begin{document}

\maketitle

\section{Affine algebraic sets}
\label{sec:orgb543cc8}
\subsection{Affine space}
\label{sec:orga7eba34}
The objects of study in algebraic geometry are called algebraic varieties.
The building blocks for general algebraic varieties are certain subsets of the affine space.
Let us first recall affine space.

Let \(k\) be a field and let \(n\) be a non-negative integer.
The \emph{affine \(n\)-space over \(k\)}, denoted by \(\mathbb A^n_k\) is the set of \(n\)-tuples \(a_1,\dots, a_n\) whose entries \(a_i\) lie in \(k\).
Thus, \(\mathbb A^n_k\) is nothing but the product \(k^n\).
The product \(k^n\) has quite a bit of extra structure---it is a \(k\)-vector space, for example---but we wish to forget it.
That is the reason for choosing different notation.
In particular, the zero tuple does not play a distinguished role.

\subsection{Affine algebraic set}
\label{sec:org8ada370}
Let \(k[x_1,\dots,x_n]\) denote the ring of polynomials in variables \(x_1, \dots, x_n\) and coefficients in \(k\).
An \emph{affine algebraic subset} of the affine space \(\mathbb A^n_k\) is the common zero locus of a set of polynomials.
More precisely, a set \(S \subset \mathbb A^n_k\) is an affine algebraic subset if there exists a set of polysomials \(A \subset k[x_1,\dots,x_n]\) such that
\[ S = \{a \in \mathbb A^n_k \mid f(a) = 0 \text{ for all } f \in A\}.\]
\subsubsection{Definition (Vanishing locus)}
\label{sec:org487ba9d}
Given \(A \subset k[x_1,\dots,x_n]\), the \emph{vanishing locus of \(A\)}, denoted by \(V(A)\) is the set 
\[ V(A) = \{a \in \mathbb A^n_k \mid f(a) = 0 \text{ for all } f \in A\}.\]

\subsubsection*{---}
\label{sec:org31e8321}
Thus the affine algebraic sets are precisely the sets of the form \(V(A)\) for some \(A\).
\subsubsection{Examples/non-examples}
\label{sec:org3356168}
The following are affine algebraic sets
\begin{enumerate}
\item The empty set
\item Entire affine space
\item Single point
\end{enumerate}
\begin{skipped}
Proof.
\end{skipped}

The following are not affine algebraic sets
\begin{enumerate}
\item The unit cube in \(\mathbb A^n_{{\mathbb R}}\)
\item Points with rational coordinates in \(\mathbb A^n_{{\mathbb C}}\)
\end{enumerate}
\begin{skipped}
Proof.
\end{skipped}

\subsection{Ideals}
\label{sec:orgf057b15}
Let \(R\) be a ring.
Recall that a subset \(I \subset R\) is an \emph{ideal} if it is closed under addition and multiplication by elements of \(R\).
Given any subset \(A \subset R\) the \emph{ideal generated by \(A\)}, denoted by \(\langle A \rangle\) is the smallest ideal containing \(A\).
This ideal consists of all elements \(r\) of  \(R\) that can be written as a linear combination
\[ r = a_1 r_1 + \cdots + a_m r_m,\]
where \(a_i \in A\) and \(r_i \in R\).

\subsubsection{Proposition}
\label{sec:org98afbe8}
Let \(A \subset k[x_1,\dots,x_n]\). Then we have \(V(A) = V(\langle A \rangle)\).
\begin{skipped}
Proof.
\end{skipped}

\subsection{Noetherian rings and the Hilbert basis theorem}
\label{sec:org2c0d170}
In our definition of \(V(A)\), the subset \(A\) may be infinite. 
But it turns out that we can replace it by a finite one without changing \(V(A)\).
This is a consequence of the Hilbert basis theorem, which, in turn, has to do with a fundamental property of rings.

We begin with a simple observation.
\subsubsection{Proposition}
\label{sec:org7fd7a14}
\label{icc}
    Let \(R\) be a ring. The following are equivalent
\begin{enumerate}
\item Every ideal of \(R\) is finitely generated.
\item Every infinite chain of ideals
\[I_1 \subset I_2 \subset I_3 \subset \cdots\]
stabilises.
\end{enumerate}

\begin{skipped}
Proof.
\end{skipped}

\subsubsection{Definition (Noetherian ring)}
\label{sec:org7feba9e}
A ring \(R\) satisfying the equivalent conditions of Proposition \ref{icc} is called \emph{Noetherian}.

\subsubsection{Examples/non-examples}
\label{sec:orgd24e9ce}
The following rings are Noetherian
\begin{enumerate}
\item \(R = {\mathbb Z}\)
\item \(R\) a field.
\end{enumerate}
\begin{skipped}
Prove this.
\end{skipped}

The ring of continuous functions on the interval is \emph{not} Noetherian.
\begin{skipped}
Prove this.
\end{skipped}

\subsubsection{Theorem}
\label{sec:orga6380c0}
If \(R\) is Noetherian, then so is \(R[x]\)

\begin{itemize}
\item Proof
\label{sec:org5337301}
Assume \(R\) is Noetherian, and let \(I \subset R[x]\) be an ideal.
We must show that \(I\) is finitely generated.
The basic idea is to use the division algorithm, while keeping track of the ideals formed by the leading coefficients.

For every non-negative integer \(m\), define
\begin{align*}
J_m &= \{{\rm Leading\ coeff} (f) \mid f \in I, f \neq 0, \quad \deg(f) \leq m\} \cup \{0\}
\end{align*}
We make the following claims.
\begin{enumerate}
\item \(J_m\) is an ideal of \(R\).
\item \(J_m \subset J_{m+1}\).
\end{enumerate}
\begin{skipped}
Prove these claims.
\end{skipped}
Since \(R\) is Noetherian, the chain \(J_1 \subset J_2 \subset \cdots\) stabilises; say \(J_m = J_{m+1} = \cdots\).
Let \(S_i\) be a finite set of generators for \(J_i\), and for \(a \in S_i\), let \(p_a \in I\) be a non-zero element of degree at most \(i\) whose leading coefficient is \(a\).
We claim that the (finite) set \(p_a \mid a \in S_1 \cup \cdots \cup S_m\) generates \(I\).    
\begin{skipped}
Prove the claim and finish the proof of the theorem.
\end{skipped}
\end{itemize}

\subsubsection{Corollary (Hilbert basis theorem)}
\label{sec:orgafe6d61}
\(k[x_1,\dots,x_n]\) is Noetherian.
\begin{skipped}
Proof.
\end{skipped}
\subsubsection{Corollary}
\label{sec:org7ec0bf8}
Every affine algebraic subset of \(\mathbb A^n_k\) is the vanishing set of a finite set of polynomials.

\begin{skipped}
Proof.
\end{skipped}

\subsection{The Zariski topology}
\label{sec:org274691d}
The notion of affine algebraic sets allows us to define a topology on \(\mathbb A^n_k\).
Recall that we can specify a topology on a set by specifying what the open subsets are, or equivalently, what the closed subsets are.
In our case, it is more convenient to do the latter.
The collection of closed subsets must satisfy the following properties.
\begin{enumerate}
\item The empty set and the entire set are closed.
\item Arbitrary intersections of closed sets are closed.
\item Finite unions of closed sets are closed.

We define the \emph{Zariski topology} on \(\mathbb A^n_k\) by setting the closed subsets to be the affine algebraic sets, namely, the sets of the form \(V(A)\) for some \(A \subset k[x_1,\dots,x_n]\).
\end{enumerate}
\subsubsection{Proposition}
\label{sec:org0b1feca}
The collection of affine algeraic subsets satisfies the three conditions above.

\begin{skipped}
Proof.
\end{skipped}

\subsubsection{Proposition}
\label{sec:orgf1568f5}
The Zariski topology on \(\mathbb A^1_k\) is the \emph{finite complement topology}.
The only closed sets are the finite sets (or the whole space).
In other words, the only open sets are the complements of finite sets (or the empty set).

\begin{skipped}
Proof.
\end{skipped}

\subsubsection{Comparison between Zariski and Euclidean topology over \(\mathbb C\).}
\label{sec:orgefc25ec}
Every Zariski closed (open) subset of \(\mathbb A^n_{\mathbb C}\) is also closed (open) in the usual Euclidean topology.
The converse is not true.
\begin{skipped}
Proof.
\end{skipped}

\subsubsection{Proposition (Polynomials are continuous)}
\label{sec:org43b57b4}
Let \(f\) be a polynomial function on \(\mathbb A^n_k\).
Then \(f\) is continuous in the Zariski topology.

\subsubsection*{---}
\label{sec:orga249adc}
The Zariski topology has very few open sets, and as a result has terrible separation properties.
It is not even Hausdorff (except in very small examples).
Nevertheless, we will see that it is extremely useful.
For one, it makes sense over every field!

\subsection{The Nullstellensatz}
\label{sec:org459e4f9}
We associated a set \(V(A)\) to a subset \(A\) of the polynomial ring \(k[x_1,\dots,x_n]\).
If we think of \(A\) as a system of equations \(\{f = 0 \mid f \in A\}\), then \(V(A)\) is the set of solutions.
We can also define a reverse operation.
The Nullstellensatz says that if \(k\) is algebraically closed, then these two operations are mutually inverse.
That is, the data of a system of equations is equivalent to the data of its set of solutions.
This pleasant fact allows us go back and forth between algebra (equations) and geometry (the solution set).

We start with a straightforward definition.
\subsubsection{Definition (Ideal vanishing on a set)}
\label{sec:org3a9d77f}
Let \(S \subset \mathbb A^n_k\) be a set.
The \emph{ideal vanishing on \(S\)}, denoted by \(I(S)\), is the set
\[ I(S) = \{f \in k[x_1,\dots,x_n] \mid f(a) = 0 \text{ for all } a \in S\}\]

\subsubsection*{---}
\label{sec:orgf30fb38}
Recall that an ideal \(I \subset k[x_1,\dots,x_n]\) is \emph{radical} if it has the property that whenever \(f^n \in I\) for some \(n > 1\), then \(f \in I\).
\subsubsection{Proposition}
\label{sec:org35d174c}
The set \(I(S)\) is a radical ideal of \(k[x_1,\dots,x_n]\).

\begin{skipped}
Proof.
\end{skipped}

\subsubsection*{---}
\label{sec:org5414699}
We now state a string of important theorems, all called the ``Nullstellensatz'', starting with the most comprehensive one.
\subsubsection{Theorem}
\label{sec:orge580319}
\label{thm:null}
Let \(k\) be an algebraically closed field. Then we have a bijection
\[ \text{Radical ideals of \(k[x_1, \dots, x_n]\)} \leftrightarrow \text{Zariski closed subsets of \(\mathbb A^n_k\)}\]
where the map from the left to the right is \(I \mapsto V(I)\) and the map from the right to the left is \(S \mapsto I(S)\).

\subsubsection{Theorem}
\label{sec:org710187d}
\label{thm:null-max}
Let \(k\) be an algebraically closed field. Then all the maximal ideals of \(k[x_1,\dots,x_n]\) are of the form \((x_1-a_1, \dots, x_2-a_n)\) for some \((a_1, \dots, a_n) \in \mathbb A^n_k\).

\begin{skipped}
Show that the statement is not true if \(k\) is not algebraically closed.
\end{skipped}

\subsubsection{Theorem}
\label{sec:orge0e1d53}
\label{thm:null-unit}
 Let \(k\) be an algebraically closed field and \(I \subset k[x_1,\dots,x_n]\) an ideal. If \(V(I) = \emptyset\), then \(I = (1)\).

\begin{skipped}
Again, show that the statement is not true if \(k\) is not algebraically closed.
\end{skipped}
Theorem \ref{thm:null-unit} says that we have a dichotomy: either a system of equations \(f_i = 0\) has a solution, or there exist polynomials \(g_i\) such that
 \[ \sum f_i g_i = 1.\]

\subsubsection{Theorem}
\label{sec:org6cc55a5}
\label{thm:null-power}
Let \(k\) be an algebraically closed field and \(I \subset k[x_1,\dots,x_n]\) an ideal. If \(f\) is identically zero on \(V(I)\), then \(f^n \in I\) for some \(n\).

\begin{skipped}
Deduce Theorems \ref{thm:null-max}, \ref{thm:null-unit}, \ref{thm:null-power} from Theorem \ref{thm:null}.
\end{skipped}

\subsection{Proof of the Nullstellensatz}
\label{sec:orgf05f9a7}
The proof of Theorem \ref{thm:null} goes via the proofs of the subsequent theorems.
We use the following result from algebra, whose proof we skip.

\subsubsection{Theorem}
\label{sec:org2288c58}
\label{thm:null-alg}
    Let \(K\) be any field and let \(L\) be a finitely generated \(K\)-algebra.
    If \(L\) is a field, then it must be a finite extension of \(K\).

\begin{itemize}
\item Proof
\label{sec:org17a2160}
See \url{https://web.ma.utexas.edu/users/allcock/expos/nullstellensatz3.pdf}
\end{itemize}


\subsubsection{Proof of Theorem \ref{thm:null-max}}
\label{sec:orgb6e39f1}
Let \(m \subset k[x_1,\dots,x_n]\) be a maximal ideal.
Taking \(K = k\) and \(L = k[x_1,\dots,x_n]/m\) in Theorem \ref{thm:null-alg}, and using that \(k\) is algebraically closed, we get that the natural map \(k \to k[x_1,\dots,x_n]/m\) is an isomorphism.
Let \(a_i \in k\) be the pre-image of \(x_i\) under this isomorphism. 
Then we have \(m = (x_1-a_1,\dots,x_n-a_n)\).
\begin{skipped}
Prove the last statement.
\end{skipped}

\subsubsection{Proof of Theorem \ref{thm:null-unit}}
\label{sec:orgafa2c2a}
Suppose \(I\) is not the unit ideal. We show that \(V(I)\) is non-empty.
To do so, we use that every proper ideal is contained in a maximal ideal.
\begin{skipped}
Finish the proof.
\end{skipped}
\subsubsection{Proof of Theorem \ref{thm:null-power}}
\label{sec:orge56a6e9}
We consider the system \(g = 0\) for \(g \in I\) and \(f \neq 0\).
Notice that the last one is not an equation, but there is a trick that allows us to convert it into an equation.
Let \(y\) be a new variable, and consider the polynomial ring \(k[x_1,\dots,x_n,y]\).
In the bigger ring, consider the system of equations \(g = 0\) for \(g \in I\) and \(yf - 1 = 0\).
By our assumption, this system of equations has no solutions.
\begin{skipped}
Finish the proof using Theorem \ref{thm:null-unit}.
\end{skipped}

\subsubsection{Proof of Theorem \ref{thm:null}.}
\label{sec:orgaa5ed98}
We show that the maps \(I \to V(I)\) and \(S \to I(S)\) are mutual inverses.
That is, we show that \(I(V(I)) = I\) if \(I\) is a radical ideal, and \(V(I(S)) = S\) if \(S\) is a Zariski closed subset of \(\mathbb A^n_k\).

Let us first show that \(I(V(I)) = I\).
Since every element of \(I\) vanishes on \(V(I)\) by definition, we have \(I \subset I(V(I))\).
It remains to show that \(I(V(I)) \subset I\).
Let \(f \in I(V(I))\).
Then \(f\) is identically zero on \(V(I)\).
By \ref{thm:null-power}, there is some \(n\) such that \(f^n \in I\).
But \(I\) is radical, so \(f \in I\).

Let us now show that \(V(I(S)) = S\).
To do so, we observe a slight strengthening of the result we just proved: for any ideal \(I\), not necessarily radical, we have \(I(V(I)) = \sqrt{I}\). 
\begin{skipped}
Prove this strengthening.
\end{skipped}
Since \(S\) is Zariski closed, we know that \(S = V(J)\) for some ideal \(J\).
So \(I(S) = I(V(J)) = \sqrt{J}.\)
But it is easy to check that \(V(J) = V(\sqrt{J})\), and hence \(V(I(S)) = S\).
The proof of Theorem \ref{thm:null} is then complete.
\begin{skipped}
Check the ``easy to check'' fact.
\end{skipped}

\subsection{Affine and quasi-affine varieties}
\label{sec:org5880d3f}
An \emph{affine variety} is a subset of the affine space that is closed in the Zariski topology.
A \emph{quasi-affine variety} is a subset of the affine space that is locally closed in the Zariski topology.
(A locally closed subset of a topological space is a set that can be expressed as an intersection of an open set and a closed set).

\section{Regular functions and maps}
\label{sec:orga1a1152}
Throughout this section, \(k\) is an algebraically closed field.

\subsection{Regular functions}
\label{sec:orge8e2d41}
Let \(S \subset \mathbb A^n\) be a set and let \(f \colon S \to k\) be a function.
Let \(a\) be a point of \(S\).

\subsubsection{Definition (Regular function)}
\label{sec:org99d2308}
We say that \(f\) is \emph{regular} (or \emph{algebraic}) \emph{at \(a\)} if there exists a Zariski open set \(U \subset \mathbb A^n\) and polynomials \(p, q \in k[x_1,\dots,x_n]\) with \(q(a) \neq 0\) such that
\[ f \equiv p/q \text{ on } S \cap U.\]
We say that \(f\) is \emph{regular} if it is regular at all points of \(S\).

In other words, \(f\) is regular at a point \(a\) if locally around \(a\) (in the Zariski topology), \(f\) can be expessed as a ratio of two polynomials.
Although the definition of a regular function makes sense for \(S \subset \mathbb A^n\), we use it only in the context of quasi-affine varieties.

\subsubsection{Example}
\label{sec:orga46cb52}
\begin{enumerate}
\item A constant function is regular.
\item Every polynomial function is regular.
\end{enumerate}
\subsubsection{Example}
\label{sec:org321ebbd}
The following example shows that a regular function \(f\) may not have a unique global expression as a quotient \(p/q\).

  Let \(S_1 \subset \mathbb A^4\) be the locally closed subset defined by
\[ xv - yu = 0 \text{ and } y \neq 0,\]
and \(S_2 \subset \mathbb A^4\) the locally closed subset defined by
\[ xv - yu = 0 \text{ and } v \neq 0.\]
Set \(S = S_1 \cup S_2\). Then \(S \subset \mathbb A^4\) is also locally closed.
Consider the functions \(x/y\) on \(S_1\) and \(u/v\) on \(S_2\).
They agree on the overlap \(S1 \cap S_2\), and hence define a function on \(S\).
This function is regular on \(S\), but it is not clear whether it has a single expression as a quotient that works on all of \(S\) (in fact, it doesn't).

\subsubsection{Proposition}
\label{sec:org2e6a76e}
The set of regular functions on \(S\) contains the constant functions, and is closed under addition, negation, and multiplication.
In other words, it forms a sub-ring of the ring of \(k\)-valued functions on \(S\).

\begin{skipped}
Proof.
\end{skipped}

\subsubsection{Definition (Ring of regular functions)}
\label{sec:org62c8024}
We denote the ring of regular functions on \(S\) by denoted by \(k[S]\).


Recall that a \(k\)-algebra is a ring \(R\) along with a ring homomorphism \(k \to R\).
The ring \(k[S]\) is naturally a \(k\)-algebra; the homomorphism \(k \to k[S]\) is the one that sends \(c \in k\) to the constant function on \(S\) with value \(c\).

The following is an easy consequence of the definition.
\subsubsection{Proposition (Local nature of regularity)}
\label{sec:org4c8650e}
Let \(f\) be a function on \(S\), and let \(\{U_i\}\) be an open cover of \(S\).
If the restriction of \(f\) to each \(U_i\) is regular, then \(f\) is regular.

\begin{skipped}
Proof.
\end{skipped}

\subsection{Regular functions on an affine variety}
\label{sec:org7166485}
Regular functions on closed subsets of \(\mathbb A^n\) are just the polynomial functions.
\subsubsection{Proposition}
\label{sec:org6438f2f}
 \label{regular_on_affine}
Let \(X \subset \mathbb A^n\) be a Zariski closed subset.
Let \(f\) be a regular function on \(X\).
Then there exists a polynomial \(P \in k[x_1,\dots,x_n]\) such that \(P(x) = f(x)\) for all \(x \in X\).

\begin{itemize}
\item Proof
\label{sec:orge34dfd0}
By definition, we know that for every \(x \in X\), there is a Zariski open set \(U \subset X\) and polynomials \(p, q\) such that \(f = p/q\) on \(U\).
The set \(U\) and the polynomials \(p, q\) may depend on \(x\), so let us denote them by \(U_x\), \(p_x\), and  \(q_x\).
We need to combine all of these \(p\)'s and \(q\)'s and construct a single polynomial \(P\) that agrees with \(f\) for all \(x\).

This is done by a ``partition of unity'' argument.
First, let us do some preparation.
We know that \(p_x / q_x = f\) on \(U_x\), but we know nothing about \(p_x\) and \(q_x\) on the complement of \(U_x\).
Our first goal is to do a small trick due to which we may assume that both \(p_x\) and \(q_x\) are identically zero on the complement of \(U_x\).

Since \(U_x \subset X\) is open, its complement is closed.
By the definition of the Zariski topology, this means that
\[X \setminus U_x = X \cap V(A),\]
for some \(A \subset k[x_1,\dots, x_n]\).
Since \(x \in U_x\), at least one of the polynomials in \(A\) must be non-zero at \(x\).
Let \(g\) be such a polynomial, and set \(U'_x = X \cap \{g \neq 0\}\).
Then we have \(U'_x \subset U_x\).
\begin{skipped}
Explain why.
\end{skipped}
Likewise, set \(p'_x = p_x \cdot g\) and \(q'_x = q_x \cdot g\).
Then, we have \(f = p'_x/q_x\) on \(U'_x\), but we also have \(p'_x \equiv q'_x \equiv 0\) on \(X \setminus U'_x\).
Thus, we may assume from the beginning that both \(p_x\) and \(q_x\) are identically zero on the complement of \(U_x\)..

Now comes the crux of the argument.
Suppose \(X = V(I)\).
Consider the set of ``denominators'' \(\{q_x \mid x \in X\}\).
Note that the system of equations
\[ g = 0 \text{ for all } g \in I \text{ and } q_x = 0 \text{ for all } x \in X\]
has no solution.
\begin{skipped}
Why?
\end{skipped}
By the Nullstellensatz, this means that the ideal \(I + \langle q_x \mid q \in X \rangle\) is the unit ideal.
That is, we can write
\[ 1 = g + r_1q_{x_1} + \dots + r_m q_{x_m}\]
for some polynomials \(r_1, \dots, r_m\).
Take \(P = r_1p_{x_1} + \dots + r_m p_{x_m}\).
Then \(f = P\) on all of \(X\).
\begin{skipped}
Why?
\end{skipped}
\end{itemize}

\subsubsection*{----}
\label{sec:org32153de}
Let \(X \subset \mathbb A^n\) be any subset.
We have a ring homomorphism
\[ \pi \colon k[x_1,\dots,x_n] \to k[X],\]
where a polynomial \(f\) is sent to the regular function it defines on \(X\).

\subsubsection{Proposition (Ring of regular functions of an affine)}
\label{sec:org0a3f246}
Let \(X \subset \mathbb A^n\) be a closed subset.
Then the ring homomorphism \(\pi \from k[x_1,\dots,x_n] \to k[X]\) induces an isomorphism
\[ k[x_1,\dots,x_n]/I(X) \xrightarrow{\sim} k[X].\]
\begin{skipped}
Proof.
\end{skipped}

\subsubsection{Example}
\label{sec:org9d11a44}
If \(X \subset \mathbb A^n\) is not closed, the map \(\pi\) need not be surjective.
\begin{skipped}
Give an example.
\end{skipped}

\subsection{Regular maps}
\label{sec:orga3ce42e}
Regular functions play the same role in algebraic geometry as continuous functions in topology or smooth functions in differential geometry.

Consider \(X \subset \mathbb A^n\) and \(Y \subset \mathbb A^m\) and a function \(f \from X \to Y\).
Write \(f\) in coordinates as
\[ f = (f_1, \dots, f_m).\]

\subsubsection{Definition (Regular map)}
\label{sec:org4fdf04c}
We say that \(f\) is \emph{regular at a point \(a \in X\)} if all its  coordinate functions \(f_1, \dots, f_m\) are regular at \(a\).
If \(f\) is regular at all points of \(X\), then we say that it is \emph{regular}.



\subsubsection{Example (Maps to \(\mathbb A^1\))}
\label{sec:org48bd03f}
A regular map to \(\mathbb A^1\) is the same as a regular function.

\subsubsection{Example (An isomorphism)}
\label{sec:org9f56cb3}




\subsection{Proposition (Elementary properties of regular maps)}
\label{sec:org328916f}
\begin{enumerate}
\item The identity map is regular.
\item The composition of two regular maps is regular.
\end{enumerate}
\subsection{Proposition (Regular maps preserve regular functions)}
\label{sec:org241d60f}
\subsection{Corollary (Isomorphic varieties have isomorphic rings of functions)}
\label{sec:org267ba06}
\subsection{Proposition (For affines, map between rings induces map between spaces)}
\label{sec:org1e1e7f7}
\subsection{Examples}
\label{sec:org6722a9b}
\end{document}